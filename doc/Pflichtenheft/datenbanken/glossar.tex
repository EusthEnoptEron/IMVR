\usepackage{xparse}
\DeclareDocumentCommand{\newdualentry}{ O{} O{} m m m m } {
	\newglossaryentry{gls-#3}{name={#5},text={#5\glsadd{#3}},
		description={#6},#1
	}
	\newacronym[type=\acronymtype, see={[Glossary:]{gls-#3}},#2]{#3}{#4}{#5\glsadd{gls-#3}}
}


% GLOSSARY
\newglossaryentry{imvr}{name={IMVR},description={Kurz f�r ``Images \& Music in VR''. Die Applikation, die im Rahmen dieses Projekts mit Unity 5 entwickelt wird}}
\newglossaryentry{unity}{name={Unity},description={Unity ist eine Spielengine, die das einfache Entwickeln von 3D-Applikation f�r diverse Endger�te erlaubt}}
\newglossaryentry{rift}{name={Oculus Rift},description={Ein \gls{hmd} von Oculus VR}}
\newglossaryentry{leap}{name={Leap Motion},description={Ein auf Infrarotkameras basierter Handerkennungs-Sensor}}
\newglossaryentry{stereoskopie}{name={Stereoskopie},description={Die Stereoskopie ist die Wiedergabe von Bildern mit einem r�umlichen Eindruck von Tiefe, der physikalisch nicht vorhanden ist. Umgangssprachlich wird Stereoskopie f�lschlich als ``3D'' bezeichnet, obwohl es sich nur um zweidimensionale Abbildungen (2D) handelt, die einen r�umlichen Eindruck vermitteln. Normale zweidimensionale Bilder ohne Tiefeneindruck werden als monoskopisch  bezeichnet. (Quelle: Wikipedia)}}
\newglossaryentry{git}{name={Git},description={Git ist ein verteiltes Versionskontrollsystem, welches haupts�chlich dazu dient, den Sourcecode eines Projekts dynamisch zu verwalten. Anders als z.B. SVN ben�tigt Git keine Verbindung zu einem zentralen Repository, da jede Kopie vollwertig ist.}}

% ACRONYMS
\newacronym[type=\acronymtype]{hmd}{HMD}{Head-Mounted Display}
\newacronym{vr}{VR}{Virtual Reality}
%\newglossaryentry{ipd}{name={IPD},description={Beschreibt den Augenabstand.}}
\newdualentry{ipd}{IPD}{Inter-Pupillary Distance}{Beschreibt den Augenabstand und stellt ein wichtiges Mass f�r die stereoskopische Darstellung von Bildern dar}