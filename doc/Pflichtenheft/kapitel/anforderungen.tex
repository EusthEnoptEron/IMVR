\chapter{Anforderungen}

In diesem Kapitel sollen die Anforderungen, die in der Vision grob geschildert wurden, konkretisiert und aufgelistet werden.

\section{Funktionale Anforderungen}

In der folgenden Tabelle sind alle Anforderungen aufgelistet mit einer Priorit�t von 1 bis 5, wobei eine tiefere Zahl f�r eine h�here Priorit�t steht.

%\begin{table}[H]
	\begin{longtable}{p{0.06\textwidth} p{0.7\textwidth} p{0.13\textwidth}} \toprule %{p{0.13\textwidth} p{0.75\textwidth}} \toprule
		\textbf{Nr.} & \textbf{Beschreibung} & \textbf{Priorit�t} \\ \midrule
		\endhead
		\multicolumn{3}{l}{\textbf{1. Elementares}} \\ \midrule
		1.1 & Die Applikation nutzt die Oculus Rift im direkten Modus f�r die Ausgabe. & 1 \\
		1.2 & Die Applikation ist vollst�ndig mit den H�nden bedienbar. & 1 \\
		1.3 & Die Applikation erlaubt den Zugriff auf alle gefundenen, validen Dateien. & 1 \\
		1.4 & Bilder k�nnen angeschaut werden, Musik kann angeh�rt werden. & 1 \\
		1.5 & Die Hintergrundmusik kann von der Musikbibliothek ausgew�hlt werden. & 2 \\
		1.6 & Die Applikation kann jederzeit beendet werden. & 1 \\
		1.7 & H�nde werden abstrakt dargestellt. & 2 \\
		1.8 & Die Applikation hat eine stetige Framerate von mindestens 60fps. & 1 \\
		\multicolumn{3}{l}{\textbf{2. �bersicht (Bilder)}} \\ \nopagebreak \midrule
		2.1 & Elemente k�nnen in einem 3D-Karussell dargestellt werden. & 1 \\
		2.2 & Elemente k�nnen selektiert werden. & 2 \\
		2.3 & Elemente k�nnen getaggt werden. & 3 \\
		2.4 & Es kann auf ein bestimmtes Element fokussiert werden (siehe Detailansicht). & 1 \\
		2.5 & Es ist eine Mehrfachselektierung m�glich. & 3 \\
		2.6 & Elemente k�nnen in anderen Anordnungen dargestellt werden (Tunnel, Fluss, Fl�che, W�rfel, etc.) & 3 \\
		2.7 & Bilder k�nnen nach Farbton, S�ttigung und Helligkeit sortiert werden. & 1 \\
		2.8 & Bilder k�nnen alphabetisch und chronologisch sortiert werden. & 2 \\
		2.9 & Bilder k�nnen nach Dateiordner gruppiert werden. & 3 \\
		\multicolumn{3}{l}{\textbf{3. Detailansicht (Bilder)}} \\ \nopagebreak \midrule
		3.1 & Bild kann vergr�ssert und verkleinert werden. & 1 \\
		3.2 & Bild kann gel�scht werden & 2 \\
		3.3 & Ein (3D)-Histogramm des Bildes wird angezeigt. & 3 \\
		3.4 & Die Metadaten von Fotos werden dargestellt & 4 \\
		3.5 & Der Hintergrund passt sich den Metadaten entsprechend an & 5 \\
		3.6 & Es k�nnen zus�tzliche Versionen des Bildes (inkl. Histogram) angezeigt werden und mit Punktoperationen ver�ndert werden. & 4 \\
		3.7 & Die zus�tzlichen Bilder k�nnen auch mit lokalen und globalen Operationen gefiltert werden. \\
		\multicolumn{3}{l}{\textbf{4. �bersicht (Musik)}} \\ \nopagebreak \midrule
		4.1 & Unterst�tzt MP3 und WAVE. & 1 \\
		4.2 & Unterst�tzt weitere Musikformate (Ogg Vorbis, FLAC, M4A, APE, TAK, etc.) & 4 \\
		4.3 & Dateien werden alphabetisch gruppiert nach Artisten dargestellt & 1 \\
		4.4 & Dateien k�nnen nach Album, Genre und Jahr gruppiert werden & 3 \\
		4.5 & Dateien k�nnen ungruppiert dargestellt werden. & 5 \\
		4.6 & Alben k�nnen anhand ihres Covers wie Bilder dargestellt werden. & 3 \\
		\multicolumn{3}{l}{\textbf{5. Detailansicht (Musik)}} \\ \midrule
		5.1 & Album mit einer Liste von Liedern wird dargestellt. & 1 \\
		5.2 & Eine Datei kann zur Wiedergabe ausgew�hlt werden. & 1 \\
		5.3 & Die Wiedergabe wird visuell untermalt (Spektrogramm) & 2 \\
		5.4 & Informationen zum Artisten werden dargestellt (Ort, Beschreibung, Gr�ndungsjahr) & 3 \\
		\multicolumn{3}{l}{\textbf{6. Indexierung}} \\ \nopagebreak \midrule
		6.1 & Der Benutzer kann einen oder mehrere Ordner angeben, die nach Bilder und Musik gescannt werden. & 1 \\
		6.2 & Die gefundenen Dateien werden mit diversen Kennwerten indexiert. & 1 \\
		6.3 & Die Indexierung kann innerhalb der Applikation durchgef�hrt werden. & 4 \\
		\multicolumn{3}{l}{\textbf{7. Sonstige Funktionen}} \\ \midrule
		7.1 & Es k�nnen Filter �ber die Musik gelegt werden (Low-Pass, High-Pass, Compressor, etc.) & 4 \\
		7.2 & Es ist eine Netzwerkfunktion vorhanden, die es erlaubt, die Fotosammlung zu zweit anzuschauen. & 5 \\
		7.3 & Bildergruppen k�nnen als "Diashow" angeschaut werden. & 4 \\
		7.4 & Gesamtstatistiken k�nnen angezeigt werden (Bildalter / Anzahl, Aufl�sung / Anzahl, Aufl�sung / Anzahl, Kartenchart mit Anzahl Fotos) & 3 \\
		\caption{Pakete}
		\label{tab:pakete}
	\end{longtable}

%\end{table}

\section{Sonstige Spezifikationen}

\begin{itemize}
	\item Die Applikation wird aufgeteilt in eine Unity-Applikation und einen Indexer.
	\item Die Unity-Applikation wird erstellt mit Unity 5 Personal.
	\item Der Indexer wird erstellt mit C\# f�r das .NET Framework 4.
	\item F�r die Speicherung der Daten wird eine SQLite-Datenbank verwendet.
	\item Als Entwicklungsumgebung wird Visual Studio 2012 benutzt.
\end{itemize}


\section{Use-Cases}

(Use-Cases mit Gesten)

\section{Designskizzen}

(Skizzen und Abl�ufe)