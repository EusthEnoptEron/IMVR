\chapter{Prototyp}

In der Inception-Phase wurde ein Prototyp geschrieben, um zu testen, ob Unity in der Lage ist, Bilder und Musik so zu manipulieren, wie es die Anforderungen verlangen.

\section{Implementierung}

Der Prototyp besteht aus zwei Teilen: dem Unity-Programm und dem Indexer, der im 
Hintergrund Infos �ber Dateien sammelt. Zwischen den beiden Prozessen gibt es eine 
dateibasierte SQLite Datenbank, auf die mit normalem SQL oder ORM-Modellen zugegriffen 
werden kann.

Um Bilddateien zu analysieren, wird ImageMagick, welches bereits eine Vielzahl von 
Kennwerten zur�ckgibt, verwendet, da dieses �ber ein praktisches C\#-Interface 
verf�gt.

Musikdateien werden noch nicht analysiert. Dies kann jedoch mithilfe des jMIR-Projekts, der VAMP-Plugins und der Echo Nest API erreicht werden. Die Visualisierung der Musik wurde mit der NAudio Library realisiert - es ist allerdings m�glich, die n�tigen Werte direkt �ber die Unity-API zu ermitteln.

Alles in allem zeigt das Unity-Programm, dass es ohne Ruckeln m�glich ist, 100 Bilder zu laden und zu 
sortieren. Die anf�nglichen +100 Draw-Calls konnten auf 2 reduziert werden.

\section{Probleme}

In Sachen Bilder gibt es ein Problem, das es zu �berwinden gilt: Unity erlaubt es nicht, 
Texturen asynchron zu laden. Texturen der Gr�sse 256x256 brauchen nach einer Optimierung 
etwa 1ms zum Laden, was f�r eine Framerate von 60fps noch verkraftbar ist. Gr�ssere 
Texturen (512, 1024) erreichen jedoch schnell eine Ladezeit von 18-50ms, was sich 
langsam aber sicher bemerkbar macht.

Eine M�glichkeit dieses Problem zu l�sen, findet sich in den nativen Plugins: Es ist in Unity 
m�glich, DirectX (oder OpenGL) Code in C++ zu schreiben. Mit dieser Methode k�nnte man evtl. 
die Texturen besser allozieren.

Einfacher w�re es allerdings, alle n�tigen Bilder einfach vorzuladen.