\chapter{Testszenarien}

Dieses Kapitel definiert eine Reihe von Tests, mit denen das Projekt gepr�ft werden kann und soll.

\section{Indexer}

\begin{description}
	\item [T1 (Anforderung 6.1)] Der Benutzer f�gt seine ``Eigenen Bilder'' und seinen Windows-Ordner zur Library hinzu.
	\item [ $\blacktriangleright$] Der Indexer durchl�uft den Ordner fehlerfrei und die Anzahl indexierter Files, die er ausgibt, stimmt. 
	\item [T2 (Anforderung 6.2)] Der Benutzer �ffnet die Applikation.
	\item [ $\blacktriangleright$] Alle hinzugef�gten Bilder werden angezeigt und sind korrekt sortiert.
	\item [T3 (Anforderung 6.3)] Der Benutzer schliesst die Applikation und l�scht den Windows-Ordner von der Library und startet den Indexer.
	\item [ $\blacktriangleright$] Die Applikation zeigt beim n�chsten Start nur noch die Dateien im Ordner ``Eigene Bilder''.
\end{description}

\section{Bilder}

\begin{description}
	\item [$\bullet$] Der Benutzer f�gt seine ``Eigenen Bilder'' zur Library hinzu.
	\item [T1 (Anforderung 2.1)] Der Benutzer �ffnet die Applikation IMVR.
		\item [ $\blacktriangleright$] Ein 3D-Karussell (oder anders 3D-Konstrukt) mit seinen Bildern wird angezeigt.
	\item [T2 (Anforderung 2.2)] Der Benutzer �ffnet das Men� und sortiert nach Helligkeit.
		\item [ $\blacktriangleright$] Die Elemente werden neu sortiert und erscheinen geordnet nach Helligkeit.
	\item [T3 (Anforderung 2.4)] Der Benutzer ber�hrt mit den H�nden eines der Bilder.
		\item [ $\blacktriangleright$] Die �bersicht verblasst und das selektierte Bild wird vergr�ssert.
	\item [T4 (Anforderung 3.1)] Der Benutzer bewegt seine H�nde auseinander entsprechend der Gestentabelle.
		\item [ $\blacktriangleright$] Das Bild vergr�ssert sich entsprechend.
	\item [T5)] Der Benutzer wischt mit der Hand entsprechend der Gestentabelle.
		\item [ $\blacktriangleright$] Die Detailansicht wird verlassen und die �bersicht kommt wieder in den Vordergrund.
\end{description}

\section{Musik}

\begin{description}
	\item [T1 (Anforderung 4.1)] Der Benutzer f�gt seine ``Eigene Musik'' zur Library hinzu und startet den Indexer.
	\item [ $\blacktriangleright$] Alle MP3 Dateien mit korrekten ID3-Tags werden laut Log erkannt.
	\item [T2 (Anforderung 4.2)] Der Benutzer �ffnet die Applikation und wechselt in den Musikmodus.
	\item [ $\blacktriangleright$] Die soeben hinzugef�gte Musik erscheint alphabetisch geordnet.
	\item [T4 (Anforderung 1.5)] Der Benutzer �ffnet das Hauptmen� und beendet die Applikation.
		\item [ $\blacktriangleright$] Die Applikation fragt einmal nach und beendet dann.
\end{description}
