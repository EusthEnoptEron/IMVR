\usepackage{xparse}
\DeclareDocumentCommand{\newdualentry}{ O{} O{} m m m m } {
	\newglossaryentry{gls-#3}{name={#5},text={#5\glsadd{#3}},
		description={#6},#1
	}
	\newacronym[type=\acronymtype, see={[Glossary:]{gls-#3}},#2]{#3}{#4}{#5\glsadd{gls-#3}}
}


% GLOSSARY
\newglossaryentry{ik}{name={IK},description={Inverse Kinematics. Das umgekehrte Aufl�sen von Gelenken anhand eines Endeffektors}}

%\newglossaryentry{imvr}{name={IMVR},description={Kurz f�r ``Images \& Music in VR''. Die Applikation, die im Rahmen dieses Projekts mit Unity 5 entwickelt wird}}
\newglossaryentry{unity}{name={Unity},description={Unity ist eine Spielengine, die das einfache Entwickeln von 3D-Applikation f�r diverse Endger�te erlaubt}}
\newglossaryentry{rift}{name={Oculus Rift},description={Ein \gls{hmd} von Oculus VR}}
\newglossaryentry{leap}{name={Leap Motion},description={Ein auf Infrarotkameras basierter Handerkennungs-Sensor}}
\newglossaryentry{stereoskopie}{name={Stereoskopie},description={Die Stereoskopie ist die Wiedergabe von Bildern mit einem r�umlichen Eindruck von Tiefe, der physikalisch nicht vorhanden ist. Umgangssprachlich wird Stereoskopie f�lschlich als ``3D'' bezeichnet, obwohl es sich nur um zweidimensionale Abbildungen (2D) handelt, die einen r�umlichen Eindruck vermitteln. Normale zweidimensionale Bilder ohne Tiefeneindruck werden als monoskopisch  bezeichnet. (Quelle: Wikipedia)}}
\newglossaryentry{git}{name={Git},description={Git ist ein verteiltes Versionskontrollsystem, welches haupts�chlich dazu dient, den Sourcecode eines Projekts dynamisch zu verwalten. Anders als z.B. SVN ben�tigt Git keine Verbindung zu einem zentralen Repository, da jede Kopie vollwertig ist.}}

% ACRONYMS
\newglossaryentry{hmd}{name={HMD}, description={Head-Mounted Display. Ein Headset, welches das Sichtfeld mit stereoskopischer Optik ersetzt.}}
\newglossaryentry{rms}{name={RMS}, description={Root Mean Square oder quadratischer Mittelwert. Gr�ssere Werte haben einen st�rkeren Einfluss als kleine. Wird mit der Wurzel des Durchschnitts der Quadratsumme berechnet}}
\newglossaryentry{imvr}{name={IMVR}, description={Images \& Music in Virtual Reality. Name der Applikation, die im Rahmen des Projektes entwickelt wurde}}
\newglossaryentry{bpm}{name={BPM}, description={Beats Per Minute. Beschreibt die Taktrate eines Liedes}}
\newglossaryentry{vcs}{name={VCS}, description={Version Control System. System zum Verwalten von Quellcode, welches es erm�glicht, jeder Zeit zu einem fr�heren Snapshot zur�ckzukehren}}
\newglossaryentry{vr}{name={VR}, description={Virtual Reality. Die Darstellung und gleichzeitige Wahrnehmung der Wirklichkeit und ihrer physikalischen Eigenschaften in einer in Echtzeit computergenerierten, interaktiven virtuellen Umgebung. Quelle: Wikipedia}}
%\newglossaryentry{ipd}{name={IPD},description={Beschreibt den Augenabstand.}}
\newglossaryentry{Fork}{name={Fork}, description={Beschreibt in der Git-Welt eine Kopie eines externen Repositories, an dem individuell vom Quellrepository weitergearbeitet werden kann}}