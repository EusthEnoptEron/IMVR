\chapter{Aufgabenstellung und Zielformulierung}

Damit klar wird, was genau die Aufgabe ist, und was zur Durchf�hrung verwendet wird, soll dieses Kapitel ein paar Fakten liefern.

\section{Aufgabenstellung}

Es soll eine Applikation namens ``\gls{imvr}'' entwickelt werden, welche Gebrauch von der Oculus Rift macht, um die Bilder- und Musiksammlung des Anwenders ansprechend darzustellen, z.B. in Form eines 3D-Karussels. Die zus�tzliche "Tiefe", die durch den Einsatz eines stereoskopischen \gls{hmd} entsteht, soll dem Anwender helfen, sich in seiner Medienbibliothek schneller zurechtzufinden.

Zus�tzlich dazu soll die Leap Motion dazu verwendet werden, um vollst�ndige Handfreiheit zu gew�hren: Der Anwender soll komplett ohne Maus und Tastatur imstande sein, sich durch seine Bilder zu navigieren.

Kurz zusammengefasst muss die Applikation:

\begin{itemize}
	\item Die Bild- und Musikbibliothek des Benutzers in stereoskopischem 3D darstellen.
	\item Diese freih�ndig durchsuchbar machen mit Sortier- und evtl. Gruppierfunktion.
	\item Die Bilder betrachtbar und die Musik abspielbar machen.
	\item Metainformationen darstellen (z.B. in Form von Diagrammen).
\end{itemize}

Zus�tzlich zur Applikation selbst soll noch ein zus�tzliches Tool entwickelt werden, welches im Voraus die Dateien auf dem Host-System indexiert und f�r die visuelle Applikation bereitstellt.

\section{Hilfsmittel \& Hardware}

Verschieden Hilfsmittel werden f�r die Durchf�hrung des Projektes gebraucht. Seitens Software sind diese:

\begin{table}[H]
	\caption{�bersicht der eingesetzten Software.}
	\centering
	\label{tab:software}
	\begin{tabular}{ l l l }
		\noalign{\smallskip} \hline \hline \noalign{\smallskip}
		\textbf{Name} & \textbf{Beschreibung} \\ \midrule
		Unity3D & Spielengine und Entwicklungsumgebung \\
		Visual Studio 2012 \& 2013 & Entwicklungsumgebung von Microsoft \\
		Blender & Open-Source 3D Modelling-Tool \\
		Krita & Open-Source Grafikbearbeitungs-Tool \\
		GIMP & Open-Source Grafikbearbeitungs-Tool \\
		LaTeX & Hilfsmittel f�r die Erstellung wissenschaftlicher Dokumente \\
		\noalign{\smallskip} \hline \noalign{\smallskip}
	\end{tabular}
\end{table}

Im Hardware-Departement wurde folgendes eingesetzt:

\begin{table}[H]
	\caption{�bersicht der eingesetzten Hardware.}
	\centering
	\label{tab:hardware}
	\begin{tabular}{ l l l }
		\noalign{\smallskip} \hline \hline \noalign{\smallskip}
		\textbf{Name} & \textbf{Beschreibung} \\ \midrule
		Oculus Rift & \gls{hmd} von Oculus VR \\
		Leap Motion & Hand- und Fingererkennungsger�t von Leap Motion, Inc. \\
		\noalign{\smallskip} \hline \noalign{\smallskip}
	\end{tabular}
\end{table}
