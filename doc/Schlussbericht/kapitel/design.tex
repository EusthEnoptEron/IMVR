\chapter{Design}

Bei der Entwicklung wurden diverse Design-Entscheidungen gef�llt, welche zum Teil bereits in einem Vorprojekt analysiert wurden. Diese sollen in diesem Kapitel aufgelistet und erl�utert werden. 

\section{System�bersicht}
\label{sec:sysoverview}
Um die Applikation zu bedienen, setzt der Anwender die Oculus Rift auf und bewegt sich im von der mitgelieferten Kamera erkennbaren Bereich.

Die Leap Motion wird prinzipiell so verwendet, wie von der Herstellerfirma vorgesehen. Das heisst, diese wird mit dem offiziellen Aufsatz \cite{leap} an der Oculus Rift befestigt, und deckt so den frontalen Sichtbereich des Anwenders ab. Dies l�sst sich gut auf Abbildung \ref{fig:systemuebersicht} erkennen.

Ebenfalls erkennbar ist, dass beide Ger�te per USB mit dem Host-Computer verbunden sind und Daten an die jeweiligen Services schicken. Diese Services werden durch die in Unity verwendeten Plugins angesteuert, und die ausgewerteten Daten in IMVR verwendet.

\afterpage {
	\begin{figure}[t!]
		\centering
		\includegraphics[width=0.8\linewidth]{bilder/systemuebersicht}
		\caption{Eine �bersicht der Technologien und wie sie verbunden sind.}
		\label{fig:systemuebersicht}
	\end{figure}
	
	\begin{table}[H]
		\centering
		\begin{tabular}{p{0.1\linewidth} p{0.3\linewidth} p{0.5\linewidth}}
			\textbf{Nr.} & \textbf{Komponente} & \textbf{Beschreibung} \\ \midrule
			1. & Oculus Rift DK2 & \gls{hmd} f�r den grafischen Output. \\
			2. & Leap Motion & Ger�t, welches H�nde erkennt und ihre Koordinaten an den Computer sendet. \\
			3. & Oculus Rift Kamera & Kamera, welche seit dem DK2 f�r das �rtliche Tracking zust�ndig ist. \\
			4. & Computer & Host-System f�r IMVR. \\
			5. & Benutzer & Benutzer, der die Oculus Rift tr�gt und mit seinen H�nden das Programm steuert. \\
		\end{tabular}
	\end{table}
}

\clearpage

\section{Systemarchitektur}

Zuerst ist es wichtig, zu verstehen, wie die Applikation grob aufgebaut ist. In Abbildung \ref{fig:flow} wird dies in zwei Schritten illustriert.

\begin{figure}[H]
\centering
\includegraphics[width=1\linewidth]{diagramme/flow}
\caption{Grober �berblick des Programmablaufs}
\label{fig:flow}
\end{figure}

Man erkennt wie der Benutzer zuerst ausserhalb der Unity-Applikation den Indexer startet und durchlaufen l�sst. Dieser durchl�uft alle Ordner, die er in einer \textit{folders.txt} findet und schreibt diese in die Datenbank. Im n�chsten Schritt werden aus diversen Quellen weitere Daten abgerufen.

Was ebenfalls auf der Grafik zu sehen ist, sind die Atlasse. Um Ressourcen zu sparen (siehe Kapitel \ref{subsec:resources}), werden alle Bilder in sogenannten Atlassen, sprich Bildersammlungen, gespeichert. Die Bilder, von denen hier die Rede ist, sind Fotos der Artisten und das Artwork der indexierten Alben. 

Sobald die Datenbank im ersten Schritt erstellt wurde, setzt der Anwender, wie in Kapitel \ref{sec:sysoverview} beschrieben, seine Oculus Rift mit dem Leap Motion Aufsatz auf und startet IMVR. Er erh�lt dann die Auswahl, welchen Modus er beschreiten will und je nach Wahl entsprechende weitere Optionen.



\section{Systemdesign}

Jetzt, wo das grobe Zusammenspiel der Elemente im System klar geworden ist, soll ein bisschen n�her auf die Unterelemente eingegangen werden. Aufgrund der Br�ckenfunktion und deshalb der globalen Relevanz, wird zuerst ein Blick auf die Klassenstruktur der Datenbank geworfen. Danach wird der Aufbau des Indexers untersucht, und zuletzt schliesslich die eigentliche Applikation.

\subsection{Commons (Datenbank)}

Um Daten zwischen den zwei Programmteilen zu transportieren, wurde noch ein weiteres Projekt erstellt mit einer eigenen Datenstruktur, welche in Abbildung \ref{fig:Indexer-Commons-1} ersichtlich ist.

\afterpage {
	\begin{figure}[t!]
		\centering
		\includegraphics[width=1\linewidth]{bilder/Indexer-IMVR-Commons.pdf}
		\caption{Klassendiagramm f�r die Klassen im IMVR.Commons Package.}
		\label{fig:Indexer-Commons-1}
	\end{figure}
	
	\begin{table}[H]
		\centering
		\begin{tabularx}{\textwidth}{ l X }
			\noalign{\smallskip} \hline \hline \noalign{\smallskip}
			\textbf{Klasse} & \textbf{Beschreibung} \\ \midrule
			Artist & Klasse f�r einen einzelnen Artisten mit Biografie und ein paar Kennwerten sowie Koordinaten und Aktivit�tsdaten. \\
			Album & Jeder Artist hat ein oder mehrere Alben, an denen er beteiligt ist. Das Album selbst hat nur ein Name und ein Jahr und kann zu mehreren K�nstlern geh�ren.\\
			Song & Repr�sentiert einen Song und geh�rt genau einem K�nstler. Jeder Song hat einen Pfad und ist somit von einer reellen Datei gest�tzt. Ein Song erh�lt zus�tzlich verschieden Kennwerte, die von Services heruntergeladen werden. \\
			TermItem & Ein Begriff (z.B. Genre), der einem K�nstler mit einem Gewicht und einer Frequenz zugeordnet wird. K�nnte noch weiter normalisiert werden, wurde aber der Einfachheit halber so gelassen.\\
			AtlasTicket & Repr�sentiert eine Position bzw. ein Bild in einem Atlas. Mit der Position ist gemeint, an welcher Stelle das Bild im Atlas erscheint.\\
			IMDB & Zentraler Zugangsknoten zu den Daten. H�lt Referenzen auf die Artisten, Songs und Atlasse. \\
			\noalign{\smallskip} \hline \noalign{\smallskip}
		\end{tabularx}
		\caption{Erkl�rung der wichtigsten Klassen}
		\label{tab:commons}
	\end{table}
}

\clearpage

Es handelt sich um ein bewusst sehr minimalistisch gehaltenes Schema, um eine Musiksammlung darzustellen. Zu den Problemen, die nicht abgedeckt werden geh�ren zum Beispiel:

\begin{itemize}
	\item Jede Ausgabe eines Liedes gilt als ein separates Lied
	\item F�r kombinierte Artisten (z.B. A feat. B) wird jeweils ein neuer Artist erstellt
	\item Mehrere Aktivit�tsperioden eines K�nstlers k�nnen nicht abgebildet werden
\end{itemize}

Es g�be keine Grenzen, wenn man eine perfekte Struktur erreichen wollte, und das liegt ausserhalb des Bereichs dieses Projektes.

\subsection{Indexer}



\subsection{IMVR}