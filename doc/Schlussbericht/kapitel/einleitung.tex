\chapter{Einleitung}
\label{ch:einleitung}

Eine der faszinierendsten Entwicklungen unserer Zeit ist zweifelsohne das Aufkommen von \gls{vr} zu einem erschwinglichen Preis. Mithilfe einer speziellen Brille, einem sogenannten \gls{hmd}, taucht der Tr�ger hierbei in eine andere Welt ein, die ihm vorgegaukelt wird. Passend zu dieser Entwicklung finden auch zunehmend neue Peripherie-Ger�te ihren Weg in den Markt. Mitunter zu den offensichtlich nat�rlichsten Methoden der Eingabe geh�ren ganz klar die eigenen zwei H�nde. Es scheint deshalb logisch, diese auch wirklich zu verwenden.

Nun geht es also darum, ein \gls{hmd}, die Oculus Rift, mit einem Hand-Sensor, der Leap Motion, zu koppeln, und ein voll funktionierendes virtuelles Erlebnis damit zu entwickeln.

In einer Vorarbeit wurde bereits gepr�ft, wie die zwei Ger�te zusammenspielen und ein gewisses \textit{Know-How} wurde erarbeitet. Die \textit{Lessons Learned} waren, dass sich die Leap Motion beispielsweise nicht wirklich eignet, um darauf ein ganzes Spiel aufzubauen und dass eine gute �berlegung zum Design der Interaktion n�tig ist.

Mit diesen Erkenntnissen im Schlepptau soll nun eine Musikapplikation entwickelt werden, welche die H�nde sinnvoll und effektiv einsetzt.