\chapter{Implementation von IMVR}

Der Indexer ist implementiert und dokumentiert, also fehlt nur noch die grafische Applikation. In diesem Kapitel wird die Implementation genauer erl�utert.

\section{Aufbau}


\section{Konzept}


\section{Interaktionskonzept}

Un�berraschenderweise findet die Interaktion des Users mit IMVR fast ausschliesslich mit seinen H�nden statt. In einer fr�hen Phase des Projektes war noch geplant, eventuell die Spracheingabe modal zu den H�nden zu gebrauchen, jedoch reichte daf�r die Zeit nicht mehr. Ein Artefakt dieses Vorhabens ist das \textit{SpeechServer}-Projekt, welches sich immer noch unter den \textit{AuxiliaryTools} befindet.



\subsection{Ringmen�}

Bei der Entwicklung von VR-Applikationen st�sst man zwingenderweise auf Situationen, in denen herk�mmliche Konzepte nicht mehr verwendet werden k�nnen. Die Platzierung und der Aufbau des Men�s ist so ein Punkt.

Es ist nicht leicht ein Men� korrekt zu platzieren. Eine statische Platzierung als Overlay h�lt das Interface zwar im sichtbaren Bereich, kann sich jedoch als "l�stig" herausstellen. L�sst man es verz�gert mitschweben, ger�t das Men� sofort ausser Kontrolle, und stellt man es irgendwo in die Szene und bel�sst es dabei, verliert man es sofort aus dem Blick.

Im Falle von IMVR bieten sich jedoch die H�nde als gut verwendbarer Ankerpunkt f�r das Men� an. Es gibt ein freies Projekt auf GitHub, welches die Finger der Hand f�r die Platzierung der Buttons in einer ringartigen Struktur verwendet. Leider befand sich das Projekt in einem zu instabilen Stadium f�r diese Arbeit, aber es lieferte die Idee f�r eine eigene, �hnliche Implementierung.

F�r IMVR wurde ebenfalls ein Ringmen� entwickelt, doch dieses verf�gt �ber keine Schaltfl�chen im herk�mmlichen Sinn. Die Finger werden auch mit Funktionen versehen, aber zum Bet�tigen benutzt der Anwender nicht seine andere Hand, sondern hebt einen Finger. Wenn ein Finger lange genug gehoben wird, wird die zugewiesene Aktion ausgef�hrt.

\subsection{Fussplatten}

Beim Erstellen einer visuellen Applikation stellt sich die Frage, wie man am besten die Struktur verdeutlichen kann. Eine Technik, die daf�r gew�hlt wurde, ist der Einsatz von "Fussplatten".

Hierbei befinden sich unter den F�ssen des Anwenders ringf�rmige Platten, welche die zwei Modi der Applikation repr�sentieren. Eine dritte, zentriert abgehobene Platte dient zur Beendigung der Applikation.

Bei diesen Platten handelt es sich um das einzige Interaktionsmittel, welches bewusst keine Eingabe durch die H�nde erfordert. In diesem Fall wird die Oculus Rift selbst als Eingabeger�t verwendet, und zwar durch den Blickwinkel.

Die Idee ist, dass der Benutzer feststellen will, ``wo er steht'', herunterschaut, und durch gehaltenen Blickkontakt mit den Fussplatten diese aktivieren kann. Entsprechend den \textit{Best Practices} wird dabei ein Indikator gef�llt, der anzeigt, wie lange der Blick noch gehalten werden muss.

Diese Art von Eingabe l�sst sich oft bei bereits erschienen Demos f�r die Oculus Rift beobachten. Das Prinzip ist sehr leicht zu implementieren und daher auch verlockend, jedoch muss mit Vorsicht vorgegangen werden: F�r den Benutzer ist es auf Dauer unangenehm, wenn von ihm st�ndige Kopfbewegungen gefordert werden. In IMVR wurde jedoch bewusst Gebrauch von dieser Methode gemacht, weil der Anwender nur selten nach unten schauen wird, und es relativ intuitiv ist.


\section{Visual Design}


\section{Herausforderungen}